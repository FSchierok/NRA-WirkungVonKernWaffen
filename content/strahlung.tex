\section{Effekte von Strahlung auf den Körper}
\begin{frame}{Strahlendosis}
\begin{itemize}
	\item wird in \si{\sievert} gemessen

\end{itemize}
\end{frame}
\begin{frame}{Strahlentypen}
	\begin{itemize}
		\item \alpha-Strahlung
		\item \beta-Strahlung
		\item \gamma-Strahlung
		\item Neutronenstrahlung
	\end{itemize}
\end{frame}
\begin{frame}{α-Strahlung}
	\begin{itemize}
		\item ${}^4_2\text{HE}^{2+}$-Kern
		\item sehr geringe Reichweite in Materie
		\item lässt sich mit wenigen \si{\milli \meter} Papier abschirmen
		\item Schädigt wenig Gewebe stark, "hohe biologische Wirksamkeit"
	\end{itemize}
\end{frame}
\begin{frame}{β-Strahlung}
\begin{itemize}
	\item $e^-$ oder $e^+$ Teilchen
	\item mittlere Reichweite in Gewebe
	\item lässt sich mit einigen \si{\milli \meter} Aluminium abschirmen
	\item geringere biologische Wirksamkeit als \alpha-Strahlung
\end{itemize}
\end{frame}
\begin{frame}{γ-Strahlung}
	\begin{itemize}
		\item hochfrequente em-Welle
		\item große Reichweite im Materie
		\item lässt sich mit einigen \si{\centi \meter} Blei abschirmen
		\item geringe biologische Wirksamkeit
	\end{itemize}
\end{frame}
\begin{frame}{Neutronenstrahlung}
\begin{itemize}
	\item Neutronen Strahl
	\item große Reichweite in Materie, Wechseltwirk hauptsächlich mit Wasser /-stoff
	\item muss erst gebremmst, dann eingefangen und die entstehende \gamma-Strahlung mit Blei abgeschrimt werden
	\item auf Grund der Wechselwirkung mit Wasser hohe biologische Wirksamkeit
\end{itemize}
\end{frame}
\nocite{BfS}
