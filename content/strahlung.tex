\section{Strahlungstypen}
\begin{frame}
	\begin{block}{Strahlungstypen}
	\end{block}
\end{frame}
\begin{frame}{effektive Strahlendosis}
\begin{itemize}
	\item Energiedosis gibt Energie pro Masse an, Einheit: Gray $[\si{\gray}]=\si{\joule \per \kilo \gram}$
	\item Sievet ist die gewichtete Energiedosis $[\si{\sievert}]=\si{\gray}$
	\item $T$ geht über alle Organe und $R$ über alle Strahlungtypen
	\item Strahlungs-Wichtungsfaktor $w_R$ gibt Schädlichkeit der Strahlung an
	\item Organdosis $H_T$ gibt die jeweils absorbierte Strahldosis an
	\item Organ-Energiedosis $D_{T,R}$ gibt die absorbierte Energie an
	\item Gewebe-Wichtungsfaktor $w_T$ gibt Anfälligkeit des Gewebes für Strahlung an
	\item $E_\mathrm{eff}=\sum_T w_TH_T =\sum_T w_T \sum_R w_RD_{T,R}$
\end{itemize}
\end{frame}
\begin{frame}{Liste aller Gewebe-Wichtungsfakoren $w_T$}
	\begin{description}
		\item[$w_T=0.20$] Keimdrüsen
		\item[$w_T=0.12$] Knochenmark, Dickdarm, Lunge, Magen
		\item[$w_T=0.05$] Blase, Brust, Leber, Speiseröhre, Schilddrüse, andere
		\item[$w_T=0.01$] Haut, Knochenoberfläche
	\end{description}
	\pdfnote{$w_T$ geschätzte Werte über Alter und Geschlecht gemittelt}
\end{frame}
\begin{frame}{Strahlentypen}
	\begin{itemize}
		\item \alpha-Strahlung
		\item \beta-Strahlung
		\item \gamma-Strahlung
		\item Neutronenstrahlung
	\end{itemize}
\end{frame}
\begin{frame}{α-Strahlung}
	\begin{itemize}
		\item He-Kern
		\item sehr geringe Reichweite in Materie
		\item lässt sich mit wenigen \si{\milli \meter} Papier abschirmen
		\item ca. \SI{50}{\micro \meter} Eindringtiefe in Gewebe
		\item Hoher biologischer Wirkungsfaktor, $w_R=20$
	\end{itemize}
	\pdfnote{\alpha Strahler verschlucken}
\end{frame}
\begin{frame}{β-Strahlung}
\begin{itemize}
	\item $e^-$ oder $e^+$ Teilchen
	\item mittlere Reichweite in Gewebe
	\item lässt sich mit einigen \si{\milli \meter} Aluminium abschirmen
	\item ca. \SI{5}{\milli \meter} Eindringtiefe in Gewebe
	\item geringe biologische Wirksamkeit, $w_R=1$
\end{itemize}
\end{frame}
\begin{frame}{γ-Strahlung}
	\begin{itemize}
		\item hochfrequente em-Welle bzw. hoch energetisches Photon
		\item große Reichweite im Materie
		\item lässt sich mit einigen \si{\centi \meter} Blei abschirmen
		\item ca. \SI{10}{\centi \meter} Halbwertsbreite in Gewebe
		\item geringe biologische Wirksamkeit $w_R=1$
	\end{itemize}
 \nocite{uni_giessen}
\end{frame}
\begin{frame}{Neutronenstrahlung}
\begin{itemize}
	\item große Reichweite in Materie, wechselwirkt hauptsächlich mit Wasser /-stoff
	\item muss erst gebremst, dann eingefangen und die entstehende \gamma-Strahlung mit Blei abgeschrimt werden
	\item auf Grund der Wechselwirkung mit Wasser hohe biologische Wirksamkeit
	\item \begin{description}
		\item[$E<\SI{10}{\kilo \electronvolt}$] $w_R=5$
		\item[$\SI{10}{\kilo \electronvolt}<E<\SI{100}{\kilo \electronvolt}$] $w_R=10$
		\item[$\SI{100}{\kilo \electronvolt}<E<\SI{2}{\mega \electronvolt}$] $w_R=20$
		\item[$\SI{2}{\mega \electronvolt}<E<\SI{20}{\mega \electronvolt}$] $w_R=10$
		\item[$\SI{20}{\mega \electronvolt}<E$] $w_R=5$
	\end{description}\cite{straschuver}
\end{itemize}
\end{frame}
\nocite{BfS}
