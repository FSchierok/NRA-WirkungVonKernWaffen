\section{Strahlungstypen}
\begin{frame}{effektive Strahlendosis}
\begin{itemize}
	\item wird in \si{\sievert} gemessen
	\item Strahlungs-Wichtungsfaktor $w_R$ gibt Schädlichkeit der Strahlung an
	\item Organ-Energiedosis $D_{T,R}$ gibt die Absorbierte Energie an
	\item Gewebe-Wichtungsfaktor $w_T$ gibt Annfälligkeit des Gewebes für Strahlung an
	\item $E_\mathrm{eff}=\sum_T w_TH_T =\sum_T w_T \sum_R w_RD_{T,R}$
	\item $T$ geht über alle Organe und $R$ über alle Strahlungtypen
\end{itemize}
\end{frame}
\begin{frame}{Strahlentypen}
	\begin{itemize}
		\item \alpha-Strahlung
		\item \beta-Strahlung
		\item \gamma-Strahlung
		\item Neutronenstrahlung
	\end{itemize}
\end{frame}
\begin{frame}{α-Strahlung}
	\begin{itemize}
		\item ${}^4_2\text{HE}^{2+}$-Kern
		\item sehr geringe Reichweite in Materie
		\item lässt sich mit wenigen \si{\milli \meter} Papier abschirmen
		\item Hoher biologischer Wirkungsfaktor, $w_R=20$
	\end{itemize}
\end{frame}
\begin{frame}{β-Strahlung}
\begin{itemize}
	\item $e^-$ oder $e^+$ Teilchen
	\item mittlere Reichweite in Gewebe
	\item lässt sich mit einigen \si{\milli \meter} Aluminium abschirmen
	\item geringe biologische Wirksamkeit, $w_R=1$
\end{itemize}
\end{frame}
\begin{frame}{γ-Strahlung}
	\begin{itemize}
		\item hochfrequente em-Welle
		\item große Reichweite im Materie
		\item lässt sich mit einigen \si{\centi \meter} Blei abschirmen
		\item geringe biologische Wirksamkeit $w_R=1$
	\end{itemize}
\end{frame}
\begin{frame}{Neutronenstrahlung}
\begin{itemize}
	\item Neutronen Strahl
	\item große Reichweite in Materie, Wechseltwirk hauptsächlich mit Wasser /-stoff
	\item muss erst gebremmst, dann eingefangen und die entstehende \gamma-Strahlung mit Blei abgeschrimt werden
	\item auf Grund der Wechselwirkung mit Wasser hohe biologische Wirksamkeit
	\item \begin{description}
		\item[$E<\SI{10}{\kilo \electronvolt}$] $w_R=5$
		\item[$\SI{10}{\kilo \electronvolt}<E<\SI{100}{\kilo \electronvolt}$] $w_R=10$
		\item[$\SI{100}{\kilo \electronvolt}<E<\SI{2}{\mega \electronvolt}$] $w_R=20$
		\item[$\SI{2}{\mega \electronvolt}<E<\SI{20}{\mega \electronvolt}$] $w_R=10$
		\item[$\SI{20}{\mega \electronvolt}<E$] $w_R=5$
	\end{description}\cite{straschuver}
\end{itemize}
\end{frame}
\nocite{BfS}
