\section{Strahlenkrankheiten}
\begin{frame}
	\begin{block}{Strahlenkrankheiten}
	\end{block}
\end{frame}
\begin{frame}{Strahlenwirkung 1}
	\begin{itemize}
		\item Knochenmarkschädigungen
		\begin{itemize}
			\item[\Rightarrow] Keine zuverlässige Blutproduktion
			\item[\Rightarrow] Abnahme der Leukozyten
		\end{itemize}
		\item Urstammzellen werden schwer beschädigt
		\begin{itemize}
			\item[\Rightarrow] temporäre oder permanente Unfruchtbarkeit
		\end{itemize}
		\item Verdauungstrakt
		\begin{itemize}
			\item[\Rightarrow] Magenschleimhäute degenerieren und unterliegendes Gewebe angegriffen
		\end{itemize}
	\end{itemize}
\end{frame}
\begin{frame}{Strahlenwirkung 2}
	\begin{itemize}
		\item Haarzellen
		\begin{itemize}
			\item[\Rightarrow] temporärer oder permanenter Haarausfall
		\end{itemize}
		\item Gefäße
		\begin{itemize}
			\item[\Rightarrow] Verlust an Elastizität und Stabilität
		\end{itemize}
		\item Haut
		\begin{itemize}
			\item[\Rightarrow] Ähnlich zu Verbrennungen, aber tiefer und langsamer
		\end{itemize}
	\end{itemize}
\end{frame}
\begin{frame}{Krebs}
	Krebsarten im Zusammenhang mit Strahlungexposition
\begin{itemize}
	\item Hautkrebs
	\item Leukämie
	\item Schilddrüsenkrebs
	\item Knochenkrebs
	\item Lungenkrebs
\end{itemize}
\end{frame}
\begin{frame}{Letale Dosis}
	Die Letale Dosis $LD_{\alpha ,t}$ gibt die Dosis an, nach der $\alpha \, \%$ der Versuchsobjekte innerhalb von $t$ Tagen gestorben sind.
	\begin{description}
		\item[$LD_{10,30}$] \SI{1}{\sievert} - \SI{2}{\sievert}
		\item[$LD_{35,30}$] \SI{2}{\sievert} - \SI{3}{\sievert}
		\item[$LD_{50,30}$] \SI{3}{\sievert} - \SI{4}{\sievert}
		\item[$LD_{60,30}$] \SI{4}{\sievert} - \SI{6}{\sievert}
		\item[$LD_{100,14}$] \SI{6}{\sievert} - \SI{10}{\sievert}
		\item[$LD_{100,7}$] \SI{10}{\sievert} - \SI{20}{\sievert}
		\item[$LD_{100,3}$] \SI{20}{\sievert} - \SI{50}{\sievert}
	\end{description}
	\pdfnote{Basierend auf Röntgen und Gamma Experimenten und gemittelt nach A-Bombe}
\end{frame}
\begin{frame}{Symptome}
	\begin{table}
	\caption{Symptome einer Strahlenkrankheit nach verschiedenen Dosen am ersten Tag.\cite{AnnuRev18_2}}
	\begin{tabular}{lll}
		\toprule
 		\SI{200}{\sievert} & \SI{20}{\sievert}& \SI{4}{\sievert}\\
		\midrule
 			 Übelkeit, Erbrechen, 				& Übelkeit, Erbrechen,& Übelkeit, Erbrechen,\\
		 	 Durchfall,  Kopfschmerzen, 	& Durchfall  					& Durchfall						\\
		 	 Hautrötung, Desorientierung,	&\phantom{Hautrötung, Desorientierung,} 										&											\\
		 	 Unruhr, Ataxie,							& 										&\phantom{Hautrötung, Desorientierung,}  										\\
		 	 Schwäche, Schläfrigkeit, 		&  										&  										\\
			 Koma, Krämpfe,								&											&											\\
			 Schock, Tod,									&											&											\\
	\end{tabular}
\end{table}

\end{frame}
\begin{frame}{Symptome}
	\begin{table}
	\caption{Symptome einer Strahlenkrankheit nach verschiedenen Dosen in der zweiten Woche.\cite{AnnuRev18_2}}
	\begin{tabular}{lll}
		\toprule
 		\SI{200}{\sievert} & \SI{20}{\sievert}& \SI{4}{\sievert}\\
		\midrule
		\phantom{Hautrötung, Desorientierung,} & Übelkeit, Erbrechen, 	&	\\
		& Durchfall, Fieber, 		&	\phantom{Hautrötung, Desorientierung,} \\
		& Hautrötung, Abmagern,	&	\\
		& Erschöpfung, Tod			&	\\
	\end{tabular}
\end{table}

\end{frame}
\begin{frame}{Symptome}
	\begin{table}
	\caption{Symptome einer Strahlenkrankheit nach verschiedenen Dosen in der dritten und vierten Woche.\cite{AnnuRev18_2}}
		\begin{tabular}{lll}
			\toprule
	 		\SI{200}{\sievert} & \SI{20}{\sievert}& \SI{4}{\sievert}\\
			\midrule
			\phantom{Hautrötung, Desorientierung,} & & Schwäche, Erschöpfung, \\
			&\phantom{Hautrötung, Desorientierung,}  & Appetitlosigkeit, Überlkeit,\\
			& & Erbrechen, Durchfall,\\
			& & Fieber, Blutungen, \\
			& & Haarausfall
		\end{tabular}

\end{table}

\end{frame}
